\documentclass[11pt,a4paper,sans]{moderncv}	
\moderncvstyle{banking}
\moderncvcolor{blue}

\usepackage{etoolbox}
\makeatletter
\patchcmd{\makelettertitle}
  {\raggedright \@opening}
  {\@opening}
  {}{}
\makeatother
\usepackage[latin1]{inputenc}
\usepackage [T1] {fontenc}
% \usepackage[utf8]{inputenc}

\usepackage[scale=0.75]{geometry}

%\setlength{\hintscolumnwidth}{3cm}			% se deseja trocar a largura da coluna    

% dados pessoais
\name{Luiz}{Angioletti}
%\title{Curriculum Vit\ae}
\address{Rua Oswaldo Cruz, 195 ap 54}{13276-230}
\phone[mobile]{+55~(19)~98406~5785}
\phone[fixed]{+55~(19)~3327~5453}
\email{luiz.angioletti@gmail.com}
\social[linkedin]{br.linkedin.com/in/luizangioletti}
\extrainfo{skype: luizfelipeasoares}					% informação opcional, caso não a queria, elimine a linha
% \photo[64pt][0.4pt]{picture}					% `64pt' é a altura a que a imagem deve ser ajustada, `0.4pt' é a espessura do quadro que a contera (informe `0pt' para nenhum contorno) e `picture' é o nome do arquivo; informação opcional, caso não a queria, elimine a linha
%\quote{Alguma cita\c{c}\~ao (opcional)} 		% informação opcional, caso não a queria, elimine a linha

% para mostrar marcadores numéricos na bibliografia (por padrão não se mostram marcadores), só é útil de desejar incluir bibliografia no CV

%\makealetter
%\renewcommand*{\bibliographyitemlabel}{\@biblabel{\arabic{enumiv}}}
%makeatother

% bibliografia com várias fontes
%\usepackage{multilib}
%\newcites{book,misc}{{Livros},{Outros}}
%------------------------------------------------------------------------------------
%				conteúdo
%------------------------------------------------------------------------------------

\begin{document}
%\begin{CJK*}{UTF8}{gbsn}					% para redigir o CV em chines usando CJK

\makecvtitle

\section{Forma\c{c}\~ao Acad\^emica}
\cventry{2006 -- 2013}{Gradua\c{c}\~ao}{Universidade Federal do Par\'a}{Bel\'em}{\textit{Engenharia da Computa\c{c}\~ao}}{} % os argumentos de 3 a 6 podem permanecer em branco
%\cventry{ano -- ano}{Grau}{Institui\c{c}\~ao}{Cidade}{\textit{Curso}}{Descri\c{c}\~ao}

%\section{T\'eses de Mestrado}
%\cvitem{t\'itulo}{\emph{T\'itulo}}
%\cvitem{orientadores}{Orientadores}
%\cvitem{descrição}{Uma breve descrição da tese.}

\section{Experi\^encias}
	\subsection{Profissionais}
		\cventry{ano -- ano}{cargo}{Empregador}{Cidade}{}{Descri\c{c}\~ao geral, n\~ao mais que uma ou duas linhas.\newline{}%
		Responsabilidades especificas:%
		\begin{itemize}
			\item responsabilidade 1;
			\item responsabilidade 2, com sub-responsabilidades:
			\begin{itemize}
				\item sub-responsabilidade (a)
				\item sub-responsabilidade (b), com sub-responsabilidades (evite fazer isso!);
				\begin{itemize}
					\item sub-sub-responsabilidade i;
					\item sub-sub-responsabilidade ii;
					\item sub-sub-responsabilidade iii;
				\end{itemize}
				\item sub-responsabilidade (c);
			\end{itemize}
			\item responsabilidade 3.
		\end{itemize}}
		\cventry{ano -- ano}{cargo}{Empregador}{Cidade}{}{Descri\c{c}\~ao na linha 1.\newline{}Descri\c{c}\~ao na linha 2.}

	\subsection{Miscel\^anea}
		\cventry{ano -- ano}{cargo}{Empregador}{Cidade}{}{Descri\c{c}\~ao}
		
\section{Idiomas}
\cvitemwithcomment{Idioma 1}{nivel}{Coment\'ario}
\cvitemwithcomment{Idioma 2}{nivel}{Coment\'ario}
\cvitemwithcomment{Idioma 3}{nivel}{Coment\'ario}

\section{Conhecimentos de Informática}
\cvdoubleitem{categoria 1}{XXXX, YYY, ZZZ}{categoria 4}{XXXX, YYY, ZZZ}
\cvdoubleitem{categoria 2}{XXXX, YYY, ZZZ}{categoria 5}{XXXX, YYY, ZZZ}
\cvdoubleitem{categoria 3}{XXXX, YYY, ZZZ}{categoria 6}{XXXX, YYY, ZZZ}

\section{Interesses}
\cvitem{hobby 1}{Descrição}
\cvitem{hobby 2}{Descrição}
\cvitem{hobby 3}{Descrição}

\section{Extra 1}
\cvlistitem{Tema 1}
\cvlistitem{Tema 2}
\cvlistitem{Tema 3}

\renewcommand{\listitemsymbol}{-~}			% para modificar os símbolos nas listas

\section{Extra 2}
\cvlistdoubleitem{Tema 1}{Tema 4}
\cvlistdoubleitem{Tema 2}{Tema 5}
\cvlistdoubleitem{Tema 3}{}

% publicações tomadas de um arquivo BibTeX sem usar multibib\renewcommand*{bibliographyitemlabel}{\@biblabel{\arabic{enumiv}}}

\nocite{*}
\bibliography{plain}
\bibliography{publications}					% `publications' deve ser o nome do arquivo BibTeX

% Com publicações tomadas de um arquivo BibTeX usando o pacote multilib
%\section{Publica\c{c}\~oes}
%\nocitebook{book1,book2}
%\bibliographystylebook{plain}
%\bibliographybook{publications}              		% 'publications' deve ser o nome do arquivo BibTeX
%\nocitemisc{misc1,misc2,misc3}
%\bibliographystylemisc{plain}
%\bibliographymisc{publications}              		% 'publications' deve ser o nome do arquivo BibTeX

%\clearpage\end{CJK*}					% se você está redigindo seu currículo em chinês usando o CJK, \clearpage é requerido pelo fancyhdr para que funcione corretamente com o CJK, note que isso eliminará a numeração de página, ao deixar \lastpage como não definido

%----------------------------------------------------------------------------------------
%	Carta de Apresentação
%----------------------------------------------------------------------------------------

% para remover a carta de apresentação, remova todo esse bloco

\clearpage

\recipient{Departamento de RH}{Empresa\\123 Rua Independ\^encia\\12345-000, Cidade, Estado} % Destinatário da carta
\date{\today} % Data da carta
\opening{Prezado Sr. ou Sra.,} % Vocativo
\closing{Grato pela aten\c{c}\~ao,} % Encerramento
\enclosure[Em anexo]{curriculum vit\ae{}} % Lista de documentos anexos

\makelettertitle % Imprimir esse título de carta

\lipsum[1-3]% Texto qualquer

\makeletterclosing % Imprimir o encerramento da carta

%----------------------------------------------------------------------------------------

\end{document}

%% fim do arquivo `template-pt.tex'.