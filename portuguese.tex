\documentclass[11pt,a4paper,sans]{moderncv}	
\moderncvstyle{banking}
\moderncvcolor{blue}

\usepackage{etoolbox}
\makeatletter
\patchcmd{\makelettertitle}
  {\raggedright \@opening}
  {\@opening}
  {}{}
\makeatother
\usepackage[utf8]{inputenc}
\usepackage [T1] {fontenc}
% \usepackage[utf8]{inputenc}

\usepackage[scale=0.75]{geometry}

%\setlength{\hintscolumnwidth}{3cm}			% se deseja trocar a largura da coluna    

% dados pessoais
\name{Luiz}{Angioletti}
%\title{Curriculum Vit\ae}
\address{Rua Eng. F. Pitta Brito, 255, ap 22}{São Paulo, SP - 04753-080}
\phone[mobile]{+55~(11)~99878~0889}
\email{luiz.angioletti@gmail.com}
\social[linkedin]{br.linkedin.com/in/luizangioletti}
\extrainfo{%
		solteiro, 31 anos, skype: luizfelipeasoares}					% informação opcional, caso não a queria, elimine a linha
% \photo[64pt][0.4pt]{picture}					% `64pt' é a altura a que a imagem deve ser ajustada, `0.4pt' é a espessura do quadro que a contera (informe `0pt' para nenhum contorno) e `picture' é o nome do arquivo; informação opcional, caso não a queria, elimine a linha
%\quote{Alguma cita\c{c}\~ao (opcional)} 		% informação opcional, caso não a queria, elimine a linha

% para mostrar marcadores numéricos na bibliografia (por padrão não se mostram marcadores), só é útil de desejar incluir bibliografia no CV

%\makealetter
%\renewcommand*{\bibliographyitemlabel}{\@biblabel{\arabic{enumiv}}}
%makeatother

% bibliografia com várias fontes
%\usepackage{multilib}
%\newcites{book,misc}{{Livros},{Outros}}
%------------------------------------------------------------------------------------
%				conteúdo
%------------------------------------------------------------------------------------

\begin{document}
%\begin{CJK*}{UTF8}{gbsn}					% para redigir o CV em chines usando CJK

\makecvtitle



\section{Formação Acadêmica}
\cventry{2006 -- 2013}{Graduação}{Universidade Federal do Pará}{Belém}{\textit{Engenharia da Computação}}{} 
\cventry{2009 -- 2010}{Intercâmbio}{Universität Stuttgart}{Stuttgart}{\textit{Elektrotechnik und Informatik}}{}

%\section{T\'eses de Mestrado}
%\cvitem{t\'itulo}{\emph{T\'itulo}}
%\cvitem{orientadores}{Orientadores}
%\cvitem{descrição}{Uma breve descrição da tese.}

\section{Experiências}
	\subsection{Profissionais}
		\cventry{2014 -- atual}{Consultor de Tecnologia}{Accenture do Brasil}{São Paulo/SP}{}{Consultor em projetos de implantação de tecnologias como SAP, Oracle RMS e sistemas customizados nas indústrias de Serviços/Bens de Consumo e de Varejo. Minhas principais responsabilidades e características de trabalho são:%
			\begin{itemize}
				\item interface com clientes nacionais e multinacionais nas fases de levantamento e amadurecimento de requisitos;
				\item elaboração compartilhada de soluções com foco na necessidade do cliente;
				\item composição de especificações técnicas e funcionais; elaboração de testes de produto;
				\item aprendizagem rápida desenvolvida por exposição a ambientes da pressão constante por resultado, qualidade e pontualidade;
				\item habilidades para trabalhar em times multidisciplinares e multiculturais;
			\end{itemize}}
		\cventry{2012 -- 2013}{Analista de Suporte}{Jambu Tecnologia e Engenharia Ltda.}{Belém}{}{Oportunidade de estágio em empresa \emph{startup} de tecnologia, o que permitiu ampla experiência nas áreas de ação da empresa, como:%
		\begin{itemize}
			\item programação para administração de sistemas em Bash e Python;
			\item atendimento a clientes, presencial e remotamente;
			\item projeto, implementação e manutenção de: redes de computadores de médio porte, servidores de autenticação e controle de tráfego, servidores com serviços em WEB (autenticação, controle de versão)
			\item instrutor em curso preparatório para certificação internacional em GNU/Linux (certificação LPI);
			\item co-autor de material didático para o curso preparatório supra-citado.
		\end{itemize}}
		\cventry{2011 -- 2012}{Gerente de TI}{Laborat\'orio de Processamento de Sinais -- LaPS -- UFPA.}{Bel\'em}{}{Responsável pelo projeto e implementação de serviços, entre os quais:
		\begin{itemize}
			\item capacitação de bolsistas para as atividades de suporte;
			\item projeto, implementação e manutenção de servidores de: autenticação e arquivos, controle de versão, serviços WEB, impressão;
		\end{itemize}}

	\subsection{Voluntariado}
		\cventry{2008 -- 2013}{Co-Fundador e Tesoureiro}{Grupo Colorindo Sonhos}{Belém}{}{O grupo, tamanho médio de 15 pessoas, visa atender necessidades materiais e humanas das instituições de apoio social regularmente instaladas na região metropolitana de Belém, promovendo ações de conscientização, fóruns, atividades complementares aos atendidos e campanhas de arrecadação. Como tesoureiro fui responsável pelas finanças do grupo, bem como por repasse e aplicação de doações.}
		\cventry{2006 -- 2012}{Coordenador de Equipe de TI}{União Espírita Paraense -- UEP}{Belém}{}{A UEP promove eventos anuais de relevância estadual, que atraem por volta de 800 participantes a cada evento. Para suprir suas necessidades de TI, criou equipe de voluntários, a qual tive o prazer de coordenar. Entre minhas funções estavam: especificação e acompanhamento de desenvolvimento de sistema de inscrição on-line, processamento de inscrições e atendimento ao público.}
		
\section{Idiomas}
\cvitemwithcomment{Inglês}{fluente para escrita, leitura e conversação}{Certificado CCBEU -- Belém}
\cvitemwithcomment{Alemão}{intermediário para escrita, leitura e conversação}{Certificado pela Universität Stuttgart}
\cvitemwithcomment{Francês}{básico para escrita, leitura e conversação}{conclusão pendente}

\section{Competências}
\cvitemwithcomment{Suítes de Escritório}{conhecimentos avançados}{Processadores de Texto e Planilhas}
\cvitemwithcomment{Suítes de Escritório}{conhecimentos básicos}{Produtores de Slides e Bases de Dados}
\cvitemwithcomment{Administração de Sistemas}{GNU/Linux}{certificação LPI em andamento}
\cvitemwithcomment{Linguagens de Programação}{experiência de trabalho}{C, Java}
\cvitemwithcomment{Linguagens de Programação}{conhecimentos avançados}{Bash, Python, LaTeX, SQL}
\cvitemwithcomment{Redes de Computadores e roteamento}{conhecimentos avançados}{Redes TCP/IP}
\cvitemwithcomment{Inteligência Computacional}{conhecimentos básicos}{Redes neurais e algoritmos genéticos}


\section{Interesses}
\cvitem{Movimentos Open Source e Free Software}{identifico-me com os princípios de conhecimento aberto e construção colaborativa que difundem tecnologia e empoderamento;}
\cvitem{Métodos de Organização Pessoal}{utilizador há anos do método GTD (\emph{Getting Things Done);}}
\cvitem{Pesquisas Sociais e Ambientais}{\emph{freelancer} em transcrição de áudios para pesquisadores da área social e/ou ambiental, especificamente para questões da Amazônia}

%%----------------------------------------------------------------------------------------
%%	Carta de Apresentação
%%----------------------------------------------------------------------------------------
%
%% para remover a carta de apresentação, remova todo esse bloco
%
%\clearpage
%
%\recipient{Departamento de RH}{Empresa\\123 Rua Independ\^encia\\12345-000, Cidade, Estado} % Destinatário da carta
%\date{\today} % Data da carta
%\opening{Prezado Sr. ou Sra.,} % Vocativo
%\closing{Grato pela aten\c{c}\~ao,} % Encerramento
%\enclosure[Em anexo]{curriculum vit\ae{}} % Lista de documentos anexos
%
%\makelettertitle % Imprimir esse título de carta
%
%% \lipsum[1-3]% Texto qualquer
%
%\makeletterclosing % Imprimir o encerramento da carta
%
%%----------------------------------------------------------------------------------------

\end{document}

%% fim do arquivo `template-pt.tex'.
